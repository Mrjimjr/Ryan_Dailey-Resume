
\documentclass[10pt]{article}


% package imports
% ---------------

\usepackage[english]{babel} % for correct language and hyphenation and stuff
\usepackage{calc}           % for easier length calculations (infix notation)
\usepackage{enumitem}       % for configuring list environments
\usepackage{fancyhdr}       % for setting header and footer
\usepackage{fontspec}       % for fonts
\usepackage{geometry}       % for setting margins (\newgeometry)
\usepackage{graphicx}       % for pictures
\usepackage{microtype}      % for microtypography stuff
\usepackage{xcolor}         % for colours
\usepackage{hyperref}
\usepackage{multirow}

% margin and column widths
% ------------------------

% margins
\newgeometry{left=15mm,right=15mm,top=15mm,bottom=15mm}

% width of the gap between left and right column
\newlength{\cvcolumngapwidth}
\setlength{\cvcolumngapwidth}{0mm}

% left column width
\newlength{\cvleftcolumnwidth}
\setlength{\cvleftcolumnwidth}{60mm}

% right column width
\newlength{\cvrightcolumnwidth}
\setlength{\cvrightcolumnwidth}{\textwidth-\cvleftcolumnwidth-\cvcolumngapwidth}

% set paragraph indentation to 0, because it screws up the whole layout otherwise
\setlength{\parindent}{5mm}


% style definitions
% -----------------
% style categories explanation:
% * \cvnameXXX is used for the name;
% * \cvsectionXXX is used for section names (left column, accompanied by a horizontal rule);
% * \cvtitleXXX is used for job/education titles (right column);
% * \cvdurationXXX is used for job/education durations (left column);
% * \cvheadingXXX is used for headings (left column);
% * \cvmainXXX (and \setmainfont) is used for main text;
% * \cvruleXXX is used for the horizontal rules denoting sections.

% font families
\defaultfontfeatures{Ligatures=TeX} % reportedly a good idea, see https://tex.stackexchange.com/a/37251

\newfontfamily{\cvnamefont}{Roboto Medium}
\newfontfamily{\cvsectionfont}{Roboto Medium}
\newfontfamily{\cvtitlefont}{Roboto Regular}
\newfontfamily{\cvdurationfont}{Roboto Light Italic}
\newfontfamily{\cvheadingfont}{Roboto Regular}
\setmainfont{Roboto Light}

% colours
\definecolor{cvnamecolor}{HTML}{000000}
\definecolor{cvsectioncolor}{HTML}{000000}
\definecolor{cvtitlecolor}{HTML}{000000}
\definecolor{cvdurationcolor}{HTML}{000000}
\definecolor{cvheadingcolor}{HTML}{000000}
\definecolor{cvmaincolor}{HTML}{000000}
\definecolor{cvrulecolor}{HTML}{000000}

\color{cvmaincolor}

% styles
\newcommand{\cvnamestyle}[1]{{\Huge\cvnamefont\textcolor{cvnamecolor}{#1}}}
\newcommand{\cvsectionstyle}[1]{{\normalsize\cvsectionfont\textcolor{cvsectioncolor}{#1}}}
\newcommand{\cvtitlestyle}[1]{{\large\cvtitlefont\textcolor{cvtitlecolor}{#1}}}
\newcommand{\cvdurationstyle}[1]{{\small\cvdurationfont\textcolor{cvdurationcolor}{#1}}}
\newcommand{\cvheadingstyle}[1]{{\normalsize\cvheadingfont\textcolor{cvheadingcolor}{#1}}}


% inter-item spacing
% ------------------

% vertical space after personal info and standard CV items
\newlength{\cvafteritemskipamount}
\setlength{\cvafteritemskipamount}{2mm plus 1.25mm minus 1.25mm}

% vertical space after sections
\newlength{\cvaftersectionskipamount}
\setlength{\cvaftersectionskipamount}{2mm plus 0.5mm minus 0.5mm}

% extra vertical space to be used when a section starts with an item with a heading (e.g. in the skills section),
% so that the heading does not follow the section name too closely
\newlength{\cvbetweensectionandheadingextraskipamount}
\setlength{\cvbetweensectionandheadingextraskipamount}{1mm plus 0.25mm minus 0.25mm}


% intra-item spacing
% ------------------

% vertical space after titles

% value to be used as parskip in right column of CV items and itemsep in lists (same for both, for consistency)
\newlength{\cvparskip}
\setlength{\cvparskip}{0.5mm plus 0.125mm minus 0.125mm}

% set global list configuration (use parskip as itemsep, and no separation otherwise)
\setlist{parsep=0mm,topsep=0mm,partopsep=0mm,itemsep=\cvparskip}


% CV commands
% -----------

% creates a "personal info" CV item with the given left and right column contents, with appropriate vertical space after
% @param #1 left column content (should be the CV photo)
% @param #2 right column content (should be the name and personal info)
\newcommand{\cvpersonalinfo}[2]{
    % left and right column
    \begin{minipage}[t]{\cvleftcolumnwidth}
        \vspace{0mm} % XXX hack to align to top, see https://tex.stackexchange.com/a/11632
        \raggedleft #1
    \end{minipage}% XXX necessary comment to avoid unwanted space
    \hspace{\cvcolumngapwidth}% XXX necessary comment to avoid unwanted space
    \begin{minipage}[t]{\cvrightcolumnwidth}
        \vspace{0mm} % XXX hack to align to top, see https://tex.stackexchange.com/a/11632
        #2
    \end{minipage}

    % space after
    \vspace{\cvafteritemskipamount}
}

% typesets a name, with appropriate vertical space after
% @param #1 name text
\newcommand{\cvname}[1]{
    % name
    \cvnamestyle{#1}

    % space after
    \vspace{0mm plus 0.75mm minus 0.75mm}
}

% typesets a line of personal info beginning with an icon, with appropriate vertical space after
% @param #1 parameters for the \includegraphics command used to include the icon
% @param #2 icon filename
% @param #3 line text
\newcommand{\cvpersonalinfolinewithicon}[3]{
    % icon, vertically aligned with text (see https://tex.stackexchange.com/a/129463)
    \raisebox{.5\fontcharht\font`E-.5\height}{\includegraphics[#1]{#2}}
    % text
    #3

    % space after
    \vspace{2mm plus 0.5mm minus 0.5mm}
}

% creates a "section" CV item with the given left column content, a horizontal rule in the right column, and with
% appropriate vertical space after
% @param #1 left column content (should be the section name)
\newcommand{\cvsection}[1]{
    % left and right column
    \begin{minipage}[t]{\cvleftcolumnwidth}
        \raggedleft\cvsectionstyle{#1}
    \end{minipage}% XXX necessary comment to avoid unwanted space
    \hspace{\cvcolumngapwidth}% XXX necessary comment to avoid unwanted space
    \begin{minipage}[t]{\cvrightcolumnwidth}
        \textcolor{cvrulecolor}{\rule{\cvrightcolumnwidth}{0.3mm}}
    \end{minipage}

    % space after
    \vspace{\cvaftersectionskipamount}
}


% typesets a title, with appropriate vertical space after
% @param #1 title text
\newcommand{\cvtitle}[1]{
    % title
    \cvtitlestyle{#1}

    % space after
    \vspace{1mm plus 0.25mm minus 0.25mm}
    % XXX need to subtract cvparskip here, because it is automatically inserted after the title "paragraph"
    \vspace{-\cvparskip}
}


% header and footer
% -----------------

% set empty header and footer
\fancyhf{}
\pagestyle{fancy}
% \rhead{Ryan Dailey \thepage \hspace{1pt}}
% set header line to zero
\renewcommand{\headrulewidth}{0pt}

% =========================================
% Variables for job
\newcommand{\company}{Scale.ai}
\newcommand{\job}{Data Scientist}

% Link Colors
\hypersetup{
  colorlinks=true,
  linkcolor=blue!50!red,
  urlcolor=gray!70!black
}


% preamble end/document start
% ===========================

\begin{document}
\cvname{Ryan Dailey}
\begin{table}[h]
    \centering
    \begin{tabular}{p{0.33\textwidth}p{0.33\textwidth}p{0.33\textwidth}}
        \begin{tabular}[c]{@{}l@{}}
            \raisebox{.5\fontcharht\font`E-.5\height}{\includegraphics[width=4mm]{phone}}
            \href{tel:13177525143}{(317) 752-5143}
            \vspace{2mm plus 0.5mm minus 0.5mm}
        \end{tabular} & \begin{tabular}[c]{@{}l@{}}
            \raisebox{.5\fontcharht\font`E-.5\height}{\includegraphics[width=4mm]{linkedin}}
            \href{https://www.linkedin.com/in/rmdailey/}{linkedin.com/in/rmdailey}
        \end{tabular} & \multirow{2}{*}{\begin{tabular}[c]{@{}l@{}}
            \raisebox{.5\fontcharht\font`E-.5\height}{\includegraphics[width=4mm]{location}} 420 S Chauncey Ave., Apt. 11 \\ 
            \hspace*{4mm} West Lafayette, Indiana, 47906
        \end{tabular}} \\
        \begin{tabular}[c]{@{}l@{}}
            \raisebox{.5\fontcharht\font`E-.5\height}{\includegraphics[width=4mm]{envelop}}
            \href{mailto:ryanmdailey1@gmail.com}{ryanmdailey1@gmail.com}
        \end{tabular} & \begin{tabular}[c]{@{}l@{}}
            \raisebox{.5\fontcharht\font`E-.5\height}{\includegraphics[width=4mm]{github}}
            \href{https://github.com/Mrjimjr/}{github.com/Mrjimjr/}
        \end{tabular} & 
    \end{tabular}
\end{table}
% work experience
% ---------------

% \cvitem{
%     % \cvdurationstyle{}
% }{

\vspace{4\cvaftersectionskipamount}

\noindent I would love to put my engineering knowledge and strengths to work for \company. 

\vspace{2\cvaftersectionskipamount}

I am particularly interested in working for \company~in the \job~role because I spent my senior year of undergrad working on my own startup in the data labeling area. At that time I was working with the CAM2 research group at Purdue (\href{https://www.cam2project.net/}{CAM2 website}). During the Summer of 2017 I was working on two projects that made me realize the need for large scale data annotations for machine learning. 

The first project was collection of visual data from network cameras around the world. I had developed several techniques that would allow our research group to collect visual data from over 120,000 network cameras around the world. This project would create opportunities leverage new visual data sources for projects like emergency response. With this data and new deep learning techniques, we could potentially identify a crash that occurred at an intersection in the middle of the night and notify emergency responders immediately without any human intervention. This technology could save many lives. However, we lacked one very important thing, training data annotations. Large, publicly available object detection datasets like ImageNet and COCO would only get us so far. We would need to create our own datasets but we did not have the resources to do this at the time. 

The second thing that made me realize that there was a need for large scale reliable annotation services was an investigation into the bias present in publicly available object recognition datasets such as ImageNet and COCO. This research showed that the public datasets the are commonly used to train computer vision datasets have implicit bias. The training done on these datasets cannot be easily transferred to new applications. For example, the datasets created by my network camera data aggregation techniques had large differences from ImageNet because the subjects were not "iconic". Objects were not in the center of the frame, were often blurry and occluded, or were low resolution. In contrast, datasets used to train many "state-of-the-art" computer vision models like SSD or YOLO have "iconic" perfectly focused and high resolution images. After compleating this research, I realized that for most application-specific purposes, there is not enough annotated training data available. Furthermore, the processes used to create datasets like ImageNet took many years and thousands of man hours. I wanted to create a better solution to this problem. 

So I began to do market research about how large companies create dataset annotation for there machine learning applications. \company~was one of the first companies I found at the time. The information on your website looked very similar to the idea that I had for an annotation platform for machine learning. It was great to see that other people saw the same need. 

After a year of working on market research and developing my idea I wrote a National Science Foundation Small Business Innovation Research Grant (NSF SBIR) in order to get funding to start my company. This process was incredible and I learned so much about doing market research and building a business plan. The grant proposed using active learning to pre-process large computer vision datasets for machine learning and identify the most relevant subset of data to manually label. Despite my effort, I was not selected to receive the grant and chose to go to graduate school as a backup. 

As a master's student, I continued my work on automated data aggregation from network cameras. I created a web crawler to automatically search the internet for network camera data. During the design and development of the web crawler, I gained new experience in parallelization, deep learning, and distributed computing.

I am now in my second-year and looking for new opportunities in this area. I would love to put my expertise in this area to work for \company~and learn how people are solving the problems I set out to solve two years ago. 

\vspace{\cvaftersectionskipamount}

\noindent I know I would be an asset to \company. 
% Given my experience with machine learning, open source development, product design, and leadership I know the \job role is perfect for my skill-set. 
May we meet in person to discuss the position?  

\vspace{2\cvaftersectionskipamount}

\noindent Thank you, 

\vspace{\cvaftersectionskipamount}

\includegraphics[height=30pt]{Signature.pdf}

Ryan Dailey 
% }


\end{document}