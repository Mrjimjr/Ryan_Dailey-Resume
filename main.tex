
\documentclass[10pt]{article}


% package imports
% ---------------

\usepackage[english]{babel} % for correct language and hyphenation and stuff
\usepackage{calc}           % for easier length calculations (infix notation)
\usepackage{enumitem}       % for configuring list environments
\usepackage{fancyhdr}       % for setting header and footer
\usepackage{fontspec}       % for fonts
\usepackage{geometry}       % for setting margins (\newgeometry)
\usepackage{graphicx}       % for pictures
\usepackage{microtype}      % for microtypography stuff
\usepackage{xcolor}         % for colours
\usepackage{hyperref}
\usepackage{multirow}
\usepackage{datatool}

% margin and column widths
% ------------------------

% margins
\newgeometry{left=10mm,right=10mm,top=15mm,bottom=10mm}

% width of the gap between left and right column
\newlength{\cvcolumngapwidth}
\setlength{\cvcolumngapwidth}{3.5mm}

% left column width
\newlength{\cvleftcolumnwidth}
\setlength{\cvleftcolumnwidth}{44.5mm}

% right column width
\newlength{\cvrightcolumnwidth}
\setlength{\cvrightcolumnwidth}{\textwidth-\cvleftcolumnwidth-\cvcolumngapwidth}

% set paragraph indentation to 0, because it screws up the whole layout otherwise
\setlength{\parindent}{0mm}


% style definitions
% -----------------
% style categories explanation:
% * \cvnameXXX is used for the name;
% * \cvsectionXXX is used for section names (left column, accompanied by a horizontal rule);
% * \cvtitleXXX is used for job/education titles (right column);
% * \cvdurationXXX is used for job/education durations (left column);
% * \cvheadingXXX is used for headings (left column);
% * \cvmainXXX (and \setmainfont) is used for main text;
% * \cvruleXXX is used for the horizontal rules denoting sections.

% font families
\defaultfontfeatures{Ligatures=TeX} % reportedly a good idea, see https://tex.stackexchange.com/a/37251

\newfontfamily{\cvnamefont}{Roboto Medium}
\newfontfamily{\cvsectionfont}{Roboto Medium}
\newfontfamily{\cvtitlefont}{Roboto Regular}
\newfontfamily{\cvdurationfont}{Roboto Light Italic}
\newfontfamily{\cvheadingfont}{Roboto Regular}
\setmainfont{Roboto Light}

% colours
\definecolor{cvnamecolor}{HTML}{000000}
\definecolor{cvsectioncolor}{HTML}{000000}
\definecolor{cvtitlecolor}{HTML}{000000}
\definecolor{cvdurationcolor}{HTML}{000000}
\definecolor{cvheadingcolor}{HTML}{000000}
\definecolor{cvmaincolor}{HTML}{000000}
\definecolor{cvrulecolor}{HTML}{000000}

\color{cvmaincolor}

% styles
\newcommand{\cvnamestyle}[1]{{\Huge\cvnamefont\textcolor{cvnamecolor}{#1}}}
\newcommand{\cvsectionstyle}[1]{{\normalsize\cvsectionfont\textcolor{cvsectioncolor}{#1}}}
\newcommand{\cvtitlestyle}[1]{{\large\cvtitlefont\textcolor{cvtitlecolor}{#1}}}
\newcommand{\cvdurationstyle}[1]{{\small\cvdurationfont\textcolor{cvdurationcolor}{#1}}}
\newcommand{\cvheadingstyle}[1]{{\normalsize\cvheadingfont\textcolor{cvheadingcolor}{#1}}}


% inter-item spacing
% ------------------

% vertical space after personal info and standard CV items
\newlength{\cvafteritemskipamount}
\setlength{\cvafteritemskipamount}{2mm plus 1.25mm minus 1.25mm}

% vertical space after sections
\newlength{\cvaftersectionskipamount}
\setlength{\cvaftersectionskipamount}{2mm plus 0.5mm minus 0.5mm}

% extra vertical space to be used when a section starts with an item with a heading (e.g. in the skills section),
% so that the heading does not follow the section name too closely
\newlength{\cvbetweensectionandheadingextraskipamount}
\setlength{\cvbetweensectionandheadingextraskipamount}{1mm plus 0.25mm minus 0.25mm}


% intra-item spacing
% ------------------

% vertical space after titles

% value to be used as parskip in right column of CV items and itemsep in lists (same for both, for consistency)
\newlength{\cvparskip}
\setlength{\cvparskip}{0.5mm plus 0.125mm minus 0.125mm}

% set global list configuration (use parskip as itemsep, and no separation otherwise)
\setlist{parsep=0mm,topsep=0mm,partopsep=0mm,itemsep=\cvparskip}


% CV commands
% -----------

% creates a "personal info" CV item with the given left and right column contents, with appropriate vertical space after
% @param #1 left column content (should be the CV photo)
% @param #2 right column content (should be the name and personal info)
\newcommand{\cvpersonalinfo}[2]{
    % left and right column
    \begin{minipage}[t]{\cvleftcolumnwidth}
        \vspace{0mm} % XXX hack to align to top, see https://tex.stackexchange.com/a/11632
        \raggedleft #1
    \end{minipage}% XXX necessary comment to avoid unwanted space
    \hspace{\cvcolumngapwidth}% XXX necessary comment to avoid unwanted space
    \begin{minipage}[t]{\cvrightcolumnwidth}
        \vspace{0mm} % XXX hack to align to top, see https://tex.stackexchange.com/a/11632
        #2
    \end{minipage}

    % space after
    \vspace{\cvafteritemskipamount}
}

% typesets a name, with appropriate vertical space after
% @param #1 name text
\newcommand{\cvname}[1]{
    % name
    \cvnamestyle{#1}

    % space after
    \vspace{0mm plus 0.75mm minus 0.75mm}
}

% typesets a line of personal info beginning with an icon, with appropriate vertical space after
% @param #1 parameters for the \includegraphics command used to include the icon
% @param #2 icon filename
% @param #3 line text
\newcommand{\cvpersonalinfolinewithicon}[3]{
    % icon, vertically aligned with text (see https://tex.stackexchange.com/a/129463)
    \raisebox{.5\fontcharht\font`E-.5\height}{\includegraphics[#1]{#2}}
    % text
    #3

    % space after
    \vspace{2mm plus 0.5mm minus 0.5mm}
}

% creates a "section" CV item with the given left column content, a horizontal rule in the right column, and with
% appropriate vertical space after
% @param #1 left column content (should be the section name)
\newcommand{\cvsection}[1]{
    % left and right column
    \begin{minipage}[t]{\cvleftcolumnwidth}
        \raggedleft\cvsectionstyle{#1}
    \end{minipage}% XXX necessary comment to avoid unwanted space
    \hspace{\cvcolumngapwidth}% XXX necessary comment to avoid unwanted space
    \begin{minipage}[t]{\cvrightcolumnwidth}
        \textcolor{cvrulecolor}{\rule{\cvrightcolumnwidth}{0.3mm}}
    \end{minipage}

    % space after
    \vspace{\cvaftersectionskipamount}
}

% creates a standard, multi-purpose CV item with the given left and right column contents, parskip set to cvparskip
% in the right column, and with appropriate vertical space after
% @param #1 left column content
% @param #2 right column content
\newcommand{\cvitem}[2]{
    % left and right column
    \begin{minipage}[t]{\cvleftcolumnwidth}
        \raggedleft #1
    \end{minipage}% XXX necessary comment to avoid unwanted space
    \hspace{\cvcolumngapwidth}% XXX necessary comment to avoid unwanted space
    \begin{minipage}[t]{\cvrightcolumnwidth}
        \setlength{\parskip}{\cvparskip} #2
    \end{minipage}

    % space after
    \vspace{\cvafteritemskipamount}
}

% typesets a title, with appropriate vertical space after
% @param #1 title text
\newcommand{\cvtitle}[1]{
    % title
    \cvtitlestyle{#1}

    % space after
    \vspace{1mm plus 0.25mm minus 0.25mm}
    % XXX need to subtract cvparskip here, because it is automatically inserted after the title "paragraph"
    \vspace{-\cvparskip}
}

% % SKILL SORTER
% % ------------
% \newcommand{\sortitem}[1]{%
%   \DTLnewrow{list}% Create a new entry
% %   \ifx#1\relax
% %     \DTLnewdbentry{list}{sortlabel}{#2}% Add entry sortlabel (no optional argument)
% %   \else
%     \DTLnewdbentry{list}{sortlabel}{#1}% Add entry sortlabel (optional argument)
% %   \fi%
%   \DTLnewdbentry{list}{description}{#1}% Add entry description
% }
% \newenvironment{skills}{%
%   \DTLifdbexists{list}{\DTLcleardb{list}}{\DTLnewdb{list}}% Create new/discard old list
% }{%
%   \DTLsort{sortlabel}{list}% Sort list
%     % \begin{itemize}%
%     %     \DTLforeach*{list}{\theDesc=description}{\theDesc}% Print each
%     % \end{itemize}
%     \DTLforeach*{list}{\theDesc=description}{\theDesc, }% Print each
% }


% header and footer
% -----------------

% set empty header and footer
\fancyhf{}
\pagestyle{fancy}
\rhead{Ryan Dailey \thepage \hspace{1pt}}
% set header line to zero
\renewcommand{\headrulewidth}{0pt}

% ===============================================================================================
% ===============================================================================================
% ===============================================================================================
% ===================================Build Settings==============================================
% ===============================================================================================
% ===============================================================================================
% ===============================================================================================
% Online vs non online

% Tell whether or not to add links
\def\online{1}
% Tell if long form
\def\cv{0}
% Tell if short form
\def\onepage{0}


% Change Spacing If Long Form 
\if\cv1
% vertical space after personal info and standard CV items
\setlength{\cvafteritemskipamount}{3mm plus 1.25mm minus 1.25mm}

% vertical space after sections
\setlength{\cvaftersectionskipamount}{3mm plus 0.5mm minus 0.5mm}

% extra vertical space to be used when a section starts with an item with a heading (e.g. in the skills section),
% so that the heading does not follow the section name too closely
\setlength{\cvbetweensectionandheadingextraskipamount}{2mm plus 0.25mm minus 0.25mm}
\fi

% Change Spacing If Short Form 
\if\onepage1
% remove header (page number)
\rhead{}
% Set new margins
\newgeometry{left=10mm,right=10mm,top=10mm,bottom=10mm}
% vertical space after personal info and standard CV items
\setlength{\cvafteritemskipamount}{1.5mm plus 1.25mm minus 1.25mm}

% vertical space after sections
\setlength{\cvaftersectionskipamount}{0.5mm plus 0.5mm minus 0.5mm}

% extra vertical space to be used when a section starts with an item with a heading (e.g. in the skills section),
% so that the heading does not follow the section name too closely
\setlength{\cvbetweensectionandheadingextraskipamount}{1mm plus 0.25mm minus 0.25mm}
\fi

\if\online1
% Link Colors
\hypersetup{
  colorlinks=true,
  linkcolor=blue!50!red,
  urlcolor=gray!70!black
}
\fi

\begin{document}
\cvname{Ryan Dailey}
\begin{table}[h]
    \centering
    \begin{tabular}{p{0.33\textwidth}p{0.33\textwidth}p{0.33\textwidth}}
        \begin{tabular}[c]{@{}l@{}}
            \raisebox{.5\fontcharht\font`E-.5\height}{\includegraphics[width=4mm]{phone}}
            \href{tel:13177525143}{(317) 752-5143}
            \vspace{2mm plus 0.5mm minus 0.5mm}
        \end{tabular} & \begin{tabular}[c]{@{}l@{}}
            \raisebox{.5\fontcharht\font`E-.5\height}{\includegraphics[width=4mm]{linkedin}}
            \href{https://www.linkedin.com/in/rmdailey/}{linkedin.com/in/rmdailey}
        \end{tabular} & \multirow{2}{*}{\begin{tabular}[c]{@{}l@{}}
            \raisebox{.5\fontcharht\font`E-.5\height}{\includegraphics[width=4mm]{location}} 1585 Anderson Rd., Apt. 430\\ 
            \hspace*{4mm} McLean, Virginia, 22102
        \end{tabular}} \\
        \begin{tabular}[c]{@{}l@{}}
            \raisebox{.5\fontcharht\font`E-.5\height}{\includegraphics[width=4mm]{envelop}}
            \href{mailto:ryanmdailey1@gmail.com}{ryanmdailey1@gmail.com}
        \end{tabular} & \begin{tabular}[c]{@{}l@{}}
            \raisebox{.5\fontcharht\font`E-.5\height}{\includegraphics[width=4mm]{github}}
            \href{https://github.com/Mrjimjr/}{github.com/Mrjimjr/}
        \end{tabular} & 
    \end{tabular}
\end{table}
% work experience
% ---------------

\cvsection{EDUCATION}

\cvitem{
    \cvdurationstyle{May 2018 -- Dec. 2019}
}{
    \cvtitle{Master's of Engineering in Computer Engineering}

    Purdue University, West Lafayette, Indiana

    \begin{itemize}[leftmargin=*]
        \item \textit{Area of Study:} Large-Scale Data Aggregation -- GPA: 3.35
        \item \textit{MS Thesis:} Automated Data Aggregation from Network Cameras
    \end{itemize}
}
\cvitem{
    \cvdurationstyle{Aug. 2014 -- May 2018}
}{
    \cvtitle{Bachelor's of Science in Computer Engineering}

    Purdue University, West Lafayette, Indiana

    \begin{itemize}[leftmargin=*]
        \item \textit{Minor:} Management -- GPA: 3.22
    \end{itemize}
}

\cvsection{ENTREPRENEURSHIP}

% master's
\cvitem{
    \cvdurationstyle{Sep. 2017 -- Oct. 2018}
}{
    \cvtitle{Founder and CEO}

    GroundTruth Inc., West Lafayette, Indiana

    \begin{itemize}[leftmargin=*]
        \item Researched techniques for smartly sampling from large datasets in order to reduce annotation overhead supervised deep learning algorithms.
        \item Wrote research grants for the National Science Foundation proposing improved techniques for data selection in application-specific deep learning model development.
        \item Conducted market research on emerging computer vision and deep learning markets.
        \item Interviewed more than 30 companies about large-scale image data collection and annotation procedures. 
    \end{itemize}
}

\cvsection{WORK EXPERIENCE}

\cvitem{
    \cvdurationstyle{Aug. 2021 -- Present}
}{
    \cvtitle{Principal Data Scientist}

    Capital One, McLean, Virginia -- Promoted from Senior Associate Data Scientist 

    \begin{itemize}[leftmargin=*]
        \item Conducted empathy interviews with stakeholders to identify shortcomings of current monitoring processes and developed new architecture standards for monitoring pipelines.
        \item Wrote design specifications for new monitoring pipeline architecture detailing core decisions and trade-offs; shared new design with stakeholders and acquired project buy-in. 
        \item Led a team of engineers to build out new monitoring architecture using open source tools (Prefect, Dask, Docker, Kubernetes) leading to more scalable, maintainable, and actionable pipelines.
    \end{itemize}
}

\cvitem{
    \cvdurationstyle{Mar. 2020 -- Aug. 2021}
}{
    \cvtitle{Senior Data Scientist}

    Capital One, McLean, Virginia

    \begin{itemize}[leftmargin=*]
        \item Investigated and implemented novel algorithms for model performance evaluation leading to reduction of risk in several key business areas.
        \item Converted proof-of-concept algorithms into an scikit-learn style Python package (testing, CICD, and documentation) that was adopted across the enterprise. 
        \item Designed and deployed (Prefect, Dask Spark) pipelines for monitoring machine learning models on millions of financial records on AWS.
        \item Deployed interactive web applications to AWS for model performance monitoring enabling generation of deeper insights for business users.
        \item Created new standard visualization package built on open-source APIs allowing users to easily collect monitoring results and convert them to PowerPoint or interactive dashboards.
        \item Served on the Intern Community Building Committee and as a Data Science Intern Mentor for two years.
    \end{itemize}
}

\cvitem{
    \cvdurationstyle{Jun. 2019 -- Aug. 2019}
}{
    \cvtitle{Data Science Intern}

    Capital One, McLean, Virginia

    \begin{itemize}[leftmargin=*]
        \item Researched new feature monitoring techniques for identifying and mitigating potential performance risks in production stage machine learning models.
        \item Conducted empathy interviews to determine customer needs and worked with the customer to ensure success criteria were met.
        \item Designed and implemented feature monitoring techniques in a Python package with extensive testing, documentation, and usage examples.
    \end{itemize}
}

\cvitem{
    \cvdurationstyle{Aug. 2018 -- Dec. 2019}
}{
    \cvtitle{Vertically Integrated Projects (VIP) Teaching Assistant}

    Purdue University, West Lafayette, Indiana

    \begin{itemize}[leftmargin=*]
        \item Created course requirements and designed assignments to help students understand and interact with real-world research in computer vision, machine learning, and data aggregation.
        \item Directed undergraduate students in development of research papers.
        \item Assessed individual student contribution and gave feedback on research problem formation and experiment design.
        \item Gave lectures on Git and GitHub, Python, Entrepreneurship and other topics. 
    \end{itemize}
}

%  SHORTEN TO ONE PAGE =====================================================
\if\cv0
\else

\cvsection{RESEARCH EXPERIENCE}

\cvitem{
    \cvdurationstyle{Dec. 2016 -- Dec. 2017}
}{
    \cvtitle{Lead Undergraduate Researcher for the CAM2 Project}

    Purdue University, West Lafayette, Indiana

    \begin{itemize}[leftmargin=*]
        \item Mentored six sub-teams of more than 30 students that conducted experiments in data collection, computer vision, network camera security, and distributed systems
        \if\online1
            (see \href{https://www.cam2project.net/}{CAM2 website})
        \fi.
        \item Maintained software and user accounts on CAM2 research lab machines.
        \item Developed data management practices and wrote extensive onboarding documentation.
    \end{itemize}
}

\cvitem{
    \cvdurationstyle{Aug. 2016 -- Dec. 2016}
}{
    \cvtitle{CAM2 Leader of Camera Database Management}

    Purdue University, West Lafayette, Indiana

    \begin{itemize}[leftmargin=*]
        \item Designed and implemented open source software and RESTful API for CAM2 camera database including a\if\online1~\href{https://github.com/PurdueCAM2Project/CameraDatabaseClient}{Python Client} and \href{https://github.com/PurdueCAM2Project/CAM2ImageArchiver}{Image Achiever}\else~Python Client and Image Achiever\fi~for downloading network camera data.
        \item Identified methods to incorporate streaming network cameras into the CAM2 database.
    \end{itemize}
}

\cvitem{
    \cvdurationstyle{May 2016 -- Aug. 2016}
}{
    \cvtitle{Undergraduate Research Assistant CAM2 Project}

    Purdue University, West Lafayette, Indiana

    \begin{itemize}[leftmargin=*]
        \item Developed novel web-scraping methods for aggregating network camera data, resulting in a database of 120,000 unique cameras in 160 countries.
        \item Created novel methodology to automatically collect frame rate, resolution, location information and other valuable metadata from network cameras.
    \end{itemize}
}

\fi

\cvsection{PUBLICATIONS}

% master's
\cvitem{
    \cvdurationstyle{}
}{
    % \if\cv
    \begin{itemize}[leftmargin=*]
    
        \item \textbf{Ryan Dailey}, Aniesh Chawla, Andrew Liu, Sripath Mishra, Ling Zhang, Josh Majors, Yung-Hsiang Lu, George K. Thiruvathukal, “\if\online1 \href{https://dl.acm.org/doi/10.1145/3450629}{Automated Discovery of Network Cameras in Heterogeneous Web Pages}\else Automated Discovery of Network Cameras in Heterogeneous Web Pages\fi,” ACM Transactions on Internet Technology, February 2022.

        \item \textbf{Ryan Dailey}, Ahmed S. Kaseb, Chandler Brown, Sam Jenkins, Sam Yellin, Fengjian Pan, Yung-Hsiang Lu, “\if\online1 \href{https://drive.google.com/open?id=1479pCURB0qsDXMOfdWBarYYTbIyrDcYf}{Creating the World’s Largest Network Camera Database}\else Creating the World’s Largest Network Camera Database\fi,” Imaging and Multimedia Analytics in a Web and Mobile World 2017.
        
        \item Kent Gauen, \textbf{Ryan Dailey}, John Laiman, Yuxiang Zi, Nirmal Asokan, Yung-Hsiang Lu, George Thiruvathukal, Mei-Ling Shyu, Shu-Ching Chen, “\if\online1 \href{https://drive.google.com/open?id=1YFEIxjftRhNtgMDuUe9-cY8GJ_6H00yZ}{Comparison of Visual Datasets for Machine Learning}\else Comparison of Visual Datasets for Machine Learning\fi” (Invited Paper), IEEE International Conference on Information Reuse 2017.
        
        \item Kent Gauen, \textbf{Ryan Dailey}, Yung-Hsiang Lu, Eunbyung Park, Wei Liu, Alexander C. Berg, Yiran Chen, “\if\online1 \href{https://drive.google.com/open?id=1ZV4mC7vhHB9v9lOCJ_r946EbLbhj4Nus}{Three Years of LowPower Image Recognition Challenge: Introduction to Special Session}\else Three Years of LowPower Image Recognition Challenge: Introduction to Special Session\fi,” Design Automation and Test in Europe 2018.
       
        \item Samira Pouyanfar, Yudong Tao, Anup Mohan, Haiman Tian, Ahmed S. Kaseb, Kent Gauen, \textbf{Ryan Dailey}, Sarah Aghajanzadeh, Yung-Hsiang Lu, Shu-Ching Chen, Mei-Ling Shyu, "\if\online1 \href{https://drive.google.com/open?id=1MIHxzYJoPLmKy7OXyZUhjhRnKTiDwypx}{Dynamic Sampling in Convolutional Neural Networks for Imbalanced Data Classification}\else Dynamic Sampling in Convolutional Neural Networks for Imbalanced Data Classification\fi", IEEE Conference on Multimedia Information Processing and Retrieval 2018.
        
        \item Yung-Hsiang Lu, George K. Thiruvathukal, Ahmed S. Kaseb, Kent Gauen, Damini Rijhwani, \textbf{Ryan Dailey}, Deeptanshu Malik, Yutong Huang, Sarah Aghajanzadeh, Minghao Guo, "\if\online1 \href{https://arxiv.org/pdf/1904.06775.pdf}{See the World through Network Cameras}\else See the World through Network Cameras\fi", Computer 52, 10 (oct 2019), 30–40.
    \end{itemize}
    % \else
    % \begin{itemize}[leftmargin=*]
    %     \item “\if\online1 \href{https://drive.google.com/open?id=1479pCURB0qsDXMOfdWBarYYTbIyrDcYf}{Creating the World’s Largest Network Camera Database}\else Creating the World’s Largest Network Camera Database\fi,” Imaging and Multimedia Analytics in a Web and Mobile World 2017.
    %     \item “\if\online1 \href{https://drive.google.com/open?id=1YFEIxjftRhNtgMDuUe9-cY8GJ_6H00yZ}{Comparison of Visual Datasets for Machine Learning}\else Comparison of Visual Datasets for Machine Learning\fi” (Invited Paper), IEEE International Conference on Information Reuse 2017.
    %     \item “\if\online1 \href{https://drive.google.com/open?id=1ZV4mC7vhHB9v9lOCJ_r946EbLbhj4Nus}{Three Years of LowPower Image Recognition Challenge: Introduction to Special Session}\else Three Years of LowPower Image Recognition Challenge: Introduction to Special Session\fi,” Design Automation and Test in Europe 2018.
    %     \item "\if\online1 \href{https://drive.google.com/open?id=1MIHxzYJoPLmKy7OXyZUhjhRnKTiDwypx}{Dynamic Sampling in Convolutional Neural Networks for Imbalanced Data Classification}\else Dynamic Sampling in Convolutional Neural Networks for Imbalanced Data Classification\fi", IEEE Conference on Multimedia Information Processing and Retrieval 2018
    % \end{itemize}
    % \fi
}

\cvsection{PATENTS}

\cvitem{
    \cvdurationstyle{}
}{
    \begin{itemize}[leftmargin=*]
        \item \if\online1 \href{https://patents.google.com/patent/US20200327174A1/en}{US Patent 11113344, “Automated Discovery of Network Camera in Heterogeneous Web Pages”}\else US Patent 11113344, “Automated Discovery of Network Camera in Heterogeneous Web Pages”\fi, Date of Patent: Sep 7, 2021, Assignee: Purdue Research Foundation. Inventors: Yung-hsiang Lu and \textbf{Ryan Merrill Dailey}. Primary Examiner: Wing F Chan, Assistant Examiner: Joseph R Maniwang.
    \end{itemize}
}


\cvsection{EXPERIENCE}

\cvitem{
    \cvdurationstyle{Jul. 2017}
}{
    \cvtitle{Low Power Image Recognition Challenge (LPIRC) – Competition Organizer}
    
    Conference on Computer Vision and Pattern Recognition (CVPR), Honolulu, Hawaii

    \begin{itemize}[leftmargin=*]
        \item Developed automated submission and scoring system for computer vision models in LPIRC competition \if\online1(\href{https://rebootingcomputing.ieee.org/lpirc}{LPIRC website})\fi.
        \item Evaluated the participants' object detection submissions and calculated final scores.
    \end{itemize}
}

\cvitem{
    \cvdurationstyle{Jan. 2017}
}{
    \cvtitle{"Creating the World’s Largest Network Camera Database" – Presenting Author}

    IS\&T Electronic Imaging 2017, San Francisco, California

    \begin{itemize}[leftmargin=*]
        \item Presented work on large-scale camera data and metadata aggregation methods published in Imaging and Multimedia Analytics in a Web and Mobile World 2017.
    \end{itemize}
}

\cvitem{
    \cvdurationstyle{Nov. 2017}
}{
    \cvtitle{"Low-Power Image Recognition Challenge (LPIRC)" – Poster}
    
    IEEE International Conference on Rebooting Computing 2017, Tysons, Virginia

    \begin{itemize}[leftmargin=*]
        \item Presented poster summarizing results of the Low Power Image Recognition Challenge.
    \end{itemize}
}

\if\cv1


\cvitem{
    \cvdurationstyle{Mar. 2019}
}{
    \cvtitle{"Automated Indexing of Visual Data from Network Cameras" – Poster}
    
    Purdue CERIAS Symposium, West Lafayette, Indiana

    \begin{itemize}[leftmargin=*]
        \item Presented general approach for automatically collecting real-time network camera feeds.
    \end{itemize}
}


\cvitem{
    \cvdurationstyle{Apr. 2017}
}{
    \cvtitle{"Analyze Visual Data from Worldwide Network Cameras" – Poster}
    
    Purdue CERIAS Symposium, West Lafayette, Indiana

    \begin{itemize}[leftmargin=*]
        \item Presented about data security, risks, and safety benefits of public network cameras.
    \end{itemize}
}


\cvitem{
    \cvdurationstyle{Jul. 2019}
}{
    \cvtitle{Attendee}
    
    Scipy, Austin, Texas

    \begin{itemize}[leftmargin=*]
        \item Attended sprints for Dask development.
    \end{itemize}
}

% \cvitem{
%     \cvdurationstyle{Oct. 2018}
% }{
%     \cvtitle{"An overview of Git and GitHub" – Guest Lecture}
    
%     Purdue University, West Lafayette, Indiana

%     \begin{itemize}[leftmargin=*]
%         \item Presented an introduction to Git and GitHub version control systems for Vertically Integrated Projects (VIP) at Purdue.
%     \end{itemize}
% }

\cvitem{
    \cvdurationstyle{Jun. 2017}
}{
    \cvtitle{"Global Network Cameras and Public Safety" – Guest Lecture}
    
    Purdue University, West Lafayette, Indiana

    \begin{itemize}[leftmargin=*]
        \item Presented "Real-Time Network Cameras and Public Safety" lecture to 2017 Morgan St. Summer Visual Analytics Workshop.
    \end{itemize}
}

\cvsection{FRATERNITY LEADERSHIP}

\cvitem{
    \cvdurationstyle{Spring and Fall 2016}
}{
    \cvtitle{Kappa Delta Rho, Theta Chapter, Purdue University}
    
    Risk Management Officer

    \begin{itemize}[leftmargin=*]
        \item Elected member of the executive board by chapter.
        \item Created and maintained a safe, sustainable environment for the members of the chapter.
        \item Communicated with other members of the executive board in planning events and enforcing the house code of conduct. 
    \end{itemize}
    
    Centurion

    \begin{itemize}[leftmargin=*]
        \item Elected by the chapter to the 3-member judiciary board.
        \item Reviewed decisions made by the executive board to ensured fair and unbiased punishment, among other responsibilities.
    \end{itemize}
}
\fi
% END ONE PAGE =================================
% skills
% ------
\cvsection{SKILLS}

\vspace{\cvbetweensectionandheadingextraskipamount}
\cvitem{
    \cvheadingstyle{Programming Languages:}
}{
    Python, JavaScript, Bash, C, C++, SQL, System Verilog, Octave, MATLAB, HTML, CSS
}
\cvitem{
    \cvheadingstyle{Libraries and Frameworks:}
}{
    AWS, Pytorch, TensorFlow, PIL, OpenCV, FFMPEG, Numpy, Pandas, Dask, Scikit Learn, Airflow, Prefect, Plotly, Spark, Hadoop, Scipy, Docker, MongoDB, Git, GitHub, Django, Flask, Apache, Jenkins, Linux, Jupyter, Anaconda, Sphinx, STM32, EagleCAD, Tableau, Node.js, React.js, Kubernetes, Helm, Skaffold
}

\end{document}